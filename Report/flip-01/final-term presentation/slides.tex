\documentclass[
 size=12pt,
 paper=smartboard, %a4paper, smartboard, screen
 mode=present, %present, handout, print
 display=slides, % slidesnotes, notes, slides
 style=tuliplab,  % TULIP Lab style
 pauseslide,
 fleqn,leqno,clock]{powerdot}

\usepackage{amssymb}
\usepackage{amsmath}
\usepackage{rotating}
\usepackage{graphicx}
\usepackage{epstopdf}
\usepackage{boxedminipage}
\usepackage{media9}
\usepackage{rotate}
\usepackage{calc}
\usepackage[absolute]{textpos}
\usepackage{psfrag,overpic}
\usepackage{fouriernc}
\usepackage{pstricks,pst-node,pst-text,pst-3d,pst-grad}
\usepackage{moreverb,epsfig,color,subfigure}
\usepackage{color}
\usepackage{pstricks}
\usepackage{pstricks-add}
\usepackage{cleveref}
\usepackage{pst-text}
\usepackage{pst-node, pst-tree}
\usepackage{booktabs}
\usepackage{etex}
\usepackage{breqn}
\usepackage{multirow}
% \usepackage{pst-rel-points}
\usepackage{listings}
\usepackage{hyperref}
\hypersetup{ % TODO: PDF meta Data
  pdftitle={Presentation Title},
  pdfauthor={Gang Li},
  pdfpagemode={FullScreen},
  pdfborder={0 0 0} 
}


% \usepackage{auto-pst-pdf}
% package to show source code

\definecolor{LightGray}{rgb}{0.9,0.9,0.9}
\newlength{\pixel}\setlength\pixel{0.000714285714\slidewidth}
\setlength{\TPHorizModule}{\slidewidth}
\setlength{\TPVertModule}{\slideheight}
\newcommand\highlight[1]{\fbox{#1}}
\newcommand\icite[1]{{\footnotesize [#1]}}

\newcommand\twotonebox[2]{\fcolorbox{pdcolor2}{pdcolor2}{#1\vphantom{#2}}\fcolorbox{pdcolor2}{white}{#2\vphantom{#1}}}
\newcommand\twotoneboxo[2]{\fcolorbox{pdcolor2}{pdcolor2}{#1}\fcolorbox{pdcolor2}{white}{#2}}
\newcommand\vpspace[1]{\vphantom{\vspace{#1}}}
\newcommand\hpspace[1]{\hphantom{\hspace{#1}}}
\newcommand\COMMENT[1]{}

\newcommand\placepos[3]{\hbox to\z@{\kern#1
        \raisebox{-#2}[\z@][\z@]{#3}\hss}\ignorespaces}


%%%%%%%%%%%%%%%%%%%%%%%%%%%%%%%%%%%%%%%%%%%%%%%%%%%%%%%%%%%%%%%%%%%%%%%%%%
%%% title
%%% TODO: Customize to your Own Title, Name, Address
%%%
\title{FLIP(01) Final-term Presentation}
\author{Rongxin Xu\\
Hunan University
% \href{mailto:gangli@acm.org}{gangli@acm.org}
% \and % more authors
}
\date{27 February 2020}



% Customize the setting of slides
\pdsetup{
% TODO: Customize the left footer, and right footer
rf={\copyright \emph{FLIP(01)}},
cf={FLIP(01) Presentation },
}


% Starts the document
\begin{document}

\maketitle

%%==========================================================================================
%%
\begin{slide}[toc=,bm=]{Outline}
  \tableofcontents[content=sections]
\end{slide}
%%
%%==========================================================================================

\section{Introduction}

\begin{slide}{Problem Description}
  \begin{itemize}
    \item
		The ubiquitousness of smartphones enables 
		people to announce an emergency they’re 
		observing in real-time. 
          \begin{itemize}
            \item
				Predict whether a real disaster has occurred 
				based on keywords, location, and Twitter text.
          \end{itemize}
  \end{itemize}
\end{slide}

\section{Data Description}
%
\begin{slide}{Attribute Information}
  \begin{itemize}
    \item<1->
          Attributes Information
    \item[1.]
          There are 3 data sets.
    \begin{description}
    	\item[train.csv] the training set.
    	\item[test.csv] the test set.
    	\item[sample\_submission.csv] a sample submission file in the correct format.
    \end{description}
    \item[2.]
          There are 3 data sets with a total of 5 attributes.
\end{itemize}
\begin{table}[htbp]
	\centering
	\caption{Attribute Information}
	\begin{tabular}{llllll}
	\hline
	% after \\: \hline or \cline{col1-col2} \cline{col3-col4} ...
	Attributes & Information                                                                            \\
	\hline
	id   & a unique identifier for each tweet                                                               \\
	text    & the text of the tweet                                                                         \\
	location     & the location the tweet was sent from (may be blank)                                      \\
	keyword     & a particular keyword from the tweet (may be blank)                                        \\
	target    &  in train.csv only, this denotes whether a tweet is about a real disaster (1) or not (0)    \\                                               \\
	\hline
	%\bottomrule
\end{tabular}
\end{table}
\end{slide}


\section{Exploratory Data Analysis}

\begin{slide}{Missing Values of keyword and location}
	Both training and test set have same ratio 
	of missing values in keyword and location.
	the fllowings ~\cref{fig:missing-values-of-keyword-and-location} are the  
	missing values of keyword and location.
	\begin{itemize}
		\item
		0.8\% of keyword is missing in both training and test set
		\item
		33\% of location is missing in both training and test set
	\end{itemize}

\begin{figure}[tbph]
	\centering
	\includegraphics[scale=0.5]{"Figures/Missing Values of Keyword and Location.eps"}
	\caption{Missing Values of Keyword and Location}
	\label{fig:missing-values-of-keyword-and-location}
\end{figure}
\end{slide}

\begin{slide}{Cardinality and Target Distribution}
The cardinality distribution is:
\begin{itemize}
	\item
	Number of unique values in keyword: 222 (Training) - 222 (Test)
	\item
	Number of unique values in location: 3342 (Training) - 1603 (Test)
\end{itemize}
The target distribution in keywords is ~\cref{fig:Target-Distribution-in-Keywords}:
\begin{figure}[tbph]
	\centering
	\includegraphics[scale=0.35]{"Figures/Target Distribution in Keywords.eps"}
	\caption{Target Distribution in Keywords}
	\label{fig:Target-Distribution-in-Keywords}
\end{figure}
\end{slide}


\begin{slide}{Meta Features}
Distributions of meta features in classes and datasets 
can be helpful to identify disaster tweets.The meta 
features used for the analysis are:

\begin{description}
	\item[word\_count] number of words in text.
	\item[unique\_word\_count] number of unique words in text.
	\item[stop\_word\_count] number of stop words in text.
	\item[url\_count] number of urls in text.
	\item[mean\_word\_length] average character count in words.
	\item[char\_count] number of characters in text.
	\item[punctuation\_count] number of punctuations in text.
	\item[hashtag\_count] number of hashtags "\#" in text.
	\item[mention\_count] number of mentions "@" in text.
\end{description}
\end{slide}

\begin{slide}{Meta Features}
\begin{figure}[tbph]
	\centering
	\includegraphics[scale=0.3]{"Figures/Meta feature target distribution1.eps"}
	\caption{Meta feature target distribution}
\end{figure}
\end{slide}

\begin{slide}{Meta Features}
	\begin{figure}[tbph]
		\centering
		\includegraphics[scale=0.3]{"Figures/Meta feature target distribution2.eps"}
		\caption{Meta feature target distribution2}
	\end{figure}
\end{slide}

\begin{slide}{Meta Features}
	\begin{figure}[tbph]
		\centering
		\includegraphics[scale=0.3]{"Figures/Meta feature target distribution3.eps"}
		\caption{Meta feature target distribution3}
	\end{figure}
\end{slide}

\begin{slide}{Target distribution}
The fllowings ~\cref{fig:Target-distribution} shows that 
class distributions are 57\% for 0 (Not Disaster) and 43\% for 1 (Disaster). 
\\
\begin{figure}[htbp]
	\centering
	\includegraphics[scale=0.5]{figures/Target distribution.eps}
	\caption{Target distribution}
	\label{fig:Target-distribution}
\end{figure}
\end{slide}

\begin{slide}{Unigrams}
Most common unigrams exist in both classes are mostly punctuations, stop words or numbers.
\begin{figure}[htbp]
	\centering
	\includegraphics[scale=0.3]{figures/Unigrams.eps}
	\caption{Unigrams}
	\label{fig:Unigrams}
\end{figure}
\end{slide}

\begin{slide}{Bigrams}
Most common bigrams in non-disaster tweets are mostly about reddit or youtube
\begin{figure}[htbp]
	\centering
	\includegraphics[scale=0.3]{figures/Bigrams.eps}
	\caption{Bigrams}
	\label{fig:Bigrams}
\end{figure}
\end{slide}

\begin{slide}{Trigrams}
	Most common bigrams in non-disaster tweets are mostly about reddit or youtube
	\begin{figure}[htbp]
		\centering
		\includegraphics[scale=0.3]{figures/Bigrams.eps}
		\caption{Tigrams}
	\end{figure}
\end{slide}

\section{Embeddings and Text Cleaning}
\begin{slide}{Embeddings Coverage}
When you have pre-trained embeddings, doing standard preprocessing 
steps  might not be a good idea because some of the valuable 
information can be lost.

Text cleaning is based on the embeddings below:

\begin{itemize}
	\item GloVe-300d-840B
	\item FastText-Crawl-300d-2M 
\end{itemize}
The fllowings ~\cref{tab:Embeddings-cover} shows that:

% Table generated by Excel2LaTeX from sheet 'Sheet1'
\begin{table}[htbp]
	\centering
	\caption{Embeddings cover}
	\begin{tabular}{lcccc}
		\toprule
		& \multicolumn{2}{c}{Training Set} & \multicolumn{2}{c}{Test Set} \\
		\midrule
		& vocabulary & text  & vocabulary & text \\
		\midrule
		Glove & 52.06\% & 82.68\% & 57.21\% & 81.85\% \\
		FastText & 51.52\% & 81.84\% & 56.55\% & 81.12\% \\
		\bottomrule
	\end{tabular}%
	\label{tab:Embeddings-cover}%
\end{table}%
\end{slide}

\begin{slide}{Text Cleaning}
The most common type of words that require cleaning in `oov` 
have punctuations at the start or end.
	
% Table generated by Excel2LaTeX from sheet 'Sheet1'
\begin{table}[htbp]
	\centering
	\caption{Text Cleaning}
	\begin{tabular}{cccc}
		\toprule
		\multicolumn{2}{c}{Special characters} & \multicolumn{2}{c}{Contractions} \\
		\midrule
		\textbackslash{}x89ÛÏWhen &  When & he's  &  he is \\
		China\textbackslash{}x89Ûªs &  China's & there's &  there is \\
		let\textbackslash{}x89Ûªs &  let's & We're &  We are \\
		fromåÊwounds &  from wounds & That's &  That is \\
		JapÌ\_n &  Japan & won't &  will not \\
		Ì©    &  e    & they're &  they are \\
		Surṳ &  Suruc & Can't &  Cannot \\
		å£3million &  3 million & wasn't &  was not \\
		\cmidrule{1-2}    \multicolumn{2}{c}{Character entity references} & don\textbackslash{}x89Ûªt &  do not \\
		\cmidrule{1-2}    \&gt  & , >   & aren't &  are not \\
		\&lt  & , <   & isn't &  is not \\
		\&amp & , \&  & What's &  What is \\
		……    & ……    & ……    & …… \\
		\bottomrule
	\end{tabular}%
\end{table}%
\end{slide}

\begin{slide}{Text Cleaning}
After text cleaning, the embedding coverage has been greatly improved.
As shown in the ~\cref{Embeddings-cover-2} below:
% Table generated by Excel2LaTeX from sheet 'Sheet1'
\begin{table}[htbp]
	\centering
	\caption{Embeddings cover 2}
	\begin{tabular}{lcccc}
		\toprule
		& \multicolumn{2}{c}{Training Set} & \multicolumn{2}{c}{Test Set} \\
		\midrule
		& vocabulary & text  & vocabulary & text \\
		\midrule
		Glove & 52.06\% & 82.68\% & 57.21\% & 81.85\% \\
		FastText & 51.52\% & 81.84\% & 56.55\% & 81.12\% \\
		\bottomrule
	\end{tabular}%
	\label{tab:Embeddings-cover-2}%
\end{table}%
\end{slide}

\section{Modeling and Model Evaluation}
\begin{slide}{BERT Layer}
	I use the implementation of BERT from the TensorFlow Models 
	repository on GitHub
  \begin{itemize}
    \item bert\_en\_uncased\_L-12\_H-768\_A-12/1
  \end{itemize}
\end{slide}

\begin{slide}{Evaluation}
The  ~\Cref{tbl:Evaluation} below shows the changes 
in the model's performance throughout the forecast process.

\begin{figure}[tbph]
	\centering
	\includegraphics[scale=0.3]{"Figures/Evaluation.eps"}
	\caption{Evaluation}
	\label{fig:Evaluation}
\end{figure}
\end{slide}

\section{Conclusion}
\begin{slide}{Conclusion}
\begin{itemize}
	\item Don't use standard preprocessing steps like stemming or 
	stopword removal when you have pre-trained embeddings.
	\item Get your vocabulary as close to the embeddings as possible.
	\item Text cleaning requires patience and meticulousness. 
	\item BERT does have an advantage over other models when 
	used properly on NLP problems.
\end{itemize}
\end{slide}


\begin{wideslide}[toc=,bm=]{}
  \centering
  \vspace{\stretch{1}}
  \twocolumn[
    lcolwidth=0.35\linewidth,
    rcolwidth=0.65\linewidth
  ]
  {
    % \centerline{\includegraphics[scale=.2]{tulip-logo.eps}}
  }
  {
    \vspace{\stretch{1}}


    \textcolor{black}{\scalebox{2.0}{Thank you \& Question}}


  }
  \vspace{\stretch{1}}
\end{wideslide}

\end{document}
\endinput
